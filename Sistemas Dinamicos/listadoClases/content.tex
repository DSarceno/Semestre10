\vspace{1cm}


\begin{table}[H]
	\centering
	\begin{tabular}{||c|p{13cm}||}
		\hline
		\hline
			No. & Temas \\
		\hline
		\hline
			1 & Programa del curso. \\
			2 & Definiciones iniciales. \\
			3 & Ejemplos y líneas de fase. \\
			4 & Propiedades de los SD. \\
			5 & Recurrencias. \\
			6 & Dinpamicas simples, introducción. \\
			7 & Topología de espacios métricos. \\
			8 & Hoja de trabajo 1. \\
			9 & Discusión y dudas. \\
			10 & Lipschitz, determinismo y contracciones. \\
			11 & El principio de contracción. \\
			12 & Ejemplo en 2D (contracción). \\
			13 & Localizando puntos fijos, y Herón. \\
			14 & Estabilidad y telarañas. \\
			15 & Examen corto 1. \\
			16 & Homeomorfismos, conjugaciones y equivalencias. \\
			17 & Invariantes, $\alpha$ y $\omega$ límites, cuencas. \\
			18 & Mapas crecientes, perturbaciones y bifurcaciones. \\
			19 & Tipos de bifurcación y tests. \\
			20 & El mapa logístico. \\
			21 & Ejercicios y dudas. \\
			22 & Dinámica en wolfram. \\
			23 & Discretización y sistemas análogos. \\
			24 & Aplicaciones de la dinámica. \\
			25 & Resolución del corto 1. \\
			26 & Variedades. \\
			27 & Diagonalización y forma de Jordan. \\
			28 & Discusión. \\
			29 & Fractales. \\
			30 & Videos 2. \\
			31 & Sistemas de EDOs. \\
			32 & Ejemplos de Sistemas. \\
			33 & Primer parcial. \\
		\hline
		\hline
	\end{tabular}
\end{table}



\begin{table}[H]
	\centering
	\begin{tabular}{||c|p{13cm}||}
		\hline
		\hline
			No. & Temas \\
		\hline
		\hline
			34 & La exponencial de una matriz. \\
			35 & Diagramas de fase 2D, parte 1. \\
			36 & Dudas de tareas. \\
			37 & Diagramas de fase 2D, parte 2. \\
			38 & Diagramas de fase 2D, parte 3. \\
			39 & Videos 3, más aplicaciones. \\
			40 & Diagrama de Poincaré, mapas lineales y logaritmos. \\
			41 & Discusión. Temas Proyectos. \\
			42 & Asignación del Examen Corto. \\
			43 & Sistemas no lineales y linealización. \\
			44 & Poincaré-Lyapunov, Hartman-Grobman y Bendixon. \\
			45 & Poincaré-Bendixon. \\
			46 & Principio de las Casillas. \\
			47 & Recurrencia y rotaciones. \\
			48 & Benford y densidad de órbitas. \\
			49 & Demostración de la densidad de órbitas. \\
			50 & Discusión. Equidistribución. \\
			51 & Demostración de la equidistribución. \\
			52 & Examen parcial 2. \\
			53 & Equidistribución vía Fourier. \\ 
			54 & Flujos y traslaciones torales. \\
			55 & Preservación de volumen para mapas. \\
			56 & Preservación de volumen para flujos. \\
			57 & Sistemas Newtonianos, introducción. \\
			58 & Mecánica de Lagrange. \\
			59 & Exposiciones 1. \\
			60 & Exposiciones 2. \\
		\hline
		\hline
	\end{tabular}
\end{table}





%%%%