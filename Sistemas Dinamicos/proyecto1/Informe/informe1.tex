\documentclass[conference]{IEEEtran}
\IEEEoverridecommandlockouts
% The preceding line is only needed to identify funding in the first footnote. If that is unneeded, please comment it out.
\usepackage{amsmath,amsthm,amssymb} %modos matemáticos y  simbolos
\usepackage{latexsym,amsfonts} %simbolos matematicos
\usepackage{cancel} %hacer la linea que cancela las ecuaciones
\usepackage[spanish, es-noshorthands]{babel} %comandos en español y cambia el cuadro por la tabla
\decimalpoint %cambia las comas por puntos decimal
\usepackage[utf8]{inputenc} %caracteristicas del español
\usepackage{physics} %Simbolos fisicos
\usepackage{array} %mejores formatos de tabla
\parindent =0cm %sangria 
\usepackage{algorithmic}
\usepackage{graphicx}
\usepackage{textcomp}
\usepackage{xcolor}
\usepackage{mathtools} 
\usepackage[framemethod=TikZ]{mdframed}%Entornos talegas
\usepackage[colorlinks = true,
			linkcolor = blue,
			citecolor = black,
			urlcolor = blue]{hyperref}%formato de los links y URL's
\usepackage{multicol} %varias columnas
\usepackage{enumerate} %enumeraciones
\usepackage{pgf,tikz,pgfplots} %documentos en formato tikz
\usepackage{mathrsfs} %letras chingonas (transformada de laplace)
\usepackage{subfigure} %varias figuras seguidas
\usepackage{tabulary}
\usepackage{multirow} %ocupar varias filas en una tabla
\usepackage{fancybox} %recuadros talegas
\usepackage{float} %ubicar graficas
\usepackage{color}
\usepackage{comment}
\usepackage{stackrel}
\usepackage{calligra}
\usepackage{lipsum}
\usepackage{cite}
\pgfplotsset{compat=1.16} 

\newcommand{\R}{\mathbb{R}}
\newcommand{\Z}{\mathbb{Z}}
%%%%%%%%%%%%%%%%%%%%%%%%%%%%%%%%%%%%%%%%%%%%%%%%%%%%%%
\def\BibTeX{{\rm B\kern-.05em{\sc i\kern-.025em b}\kern-.08em
    T\kern-.1667em\lower.7ex\hbox{E}\kern-.125emX}}
\begin{document}

\title{Salseo, El Chisme Hecho Autómata \\
{\footnotesize \scshape{Proyecto 1}}
}

\author{\IEEEauthorblockN{1\textsuperscript{st} Diego Sarceño Ramírez}
\IEEEauthorblockA{\textit{201900109} 
}
%\and
%\IEEEauthorblockN{2\textsuperscript{nd} Andrés Pérez}
%\IEEEauthorblockA{\textit{201704199}
%}
%\and
%\IEEEauthorblockN{3\textsuperscript{rd} Diego Sarceño Ramírez}
%\IEEEauthorblockA{\textit{201900109} \\
%}
}



\maketitle

\begin{abstract}

\end{abstract}

\begin{IEEEkeywords}

\end{IEEEkeywords}

\section{Objetivos}

\subsection{General}
    \begin{enumerate}[1.]
        \item 
    \end{enumerate}
\subsection{Específicos}
    \begin{enumerate}
        \item 
        \item 
        \item 
    \end{enumerate}
%\section{Introducción}
    
\section{Marco Teórico}
    
    
\section{Diseño Experimental}
    \subsection{Materiales a Utilizar}
        \begin{itemize}
        \item 2x Protoboard
    	\item 3x Pulsadores
    	\item 1x interruptor SPST (o pulsador de enclave)
    	\item 1x fuente de alimentación (ver apartado anterior con todas las alternativas)
    	\item 2x capacitores electrolíticos de 47 $\mu$F 16V
        \item 2x capacitores cerámicos de $100$nF $25$V
        \item 2x resistencias de $1$ k$\Omega$
        \item 2x LEDs verdes
        \item 2x LEDs rojos
        \item 2x LEDs amarillos
        \item 6x Resistencias $220\Omega \leq R \leq 1$k$\Omega$
        \item 2x Resistencias para temporización de reloj
    	\item 1x Capacitor para temporización de reloj
        \item 1x Capacitor $10$nF $\leq C \leq 100$nF
        \item 1x Circuito integrado temporizador 555
        \item Flip-flops de acuerdo a su diseño
        \item Las compuertas lógicas a utilizar dependen del diseño final de cada grupo (AND, OR, NOT, XOR, NAND, XNOR)
        \item 6x metros de alambre para protoboard calibre 22.
    \end{itemize}

    \subsection{Procedimientos}
        \begin{enumerate}
            \item 
            \item 
            \item 
        \end{enumerate}
\section{Resultados}
    
\section{Conclusiones}
\begin{enumerate}
    \item 
    \item 
\end{enumerate}
%\section{Recomendaciones}

\section{Anexos}
    
    
%    \begin{figure}[H]
%        \centering
%        \includegraphics[width = 0.5\textwidth]{Imagenes/DiagEstado2.png}
%        \caption{Diagrama de estados para el semáforo de la calle.}
%        \label{fig:DiagCalle}
%    \end{figure}
    
    
\begin{thebibliography}{00}
\bibitem{b1} Mano, M., 2003. \textit{Diseño Digital}. 3rd ed. México: PEARSON EDUCACIÓN.
\bibitem{b2} 2021. \textit{Circuit Diagram}. \url{https://www.circuit-diagram.org/}
\end{thebibliography}

\end{document}


