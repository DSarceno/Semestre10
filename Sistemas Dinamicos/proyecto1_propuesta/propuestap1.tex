\input{/home/diego/Documents/Licenciatura/LatexBasic/Preamble_general}

%%%%%%%%%%%%%%%%%%%%%%%%%%%%%%%%%%%%%%%%%%%%%%%%%%%%%%%%%%%
\usepackage{fancyhdr}%formato de pagina
\pagestyle{fancy}%colocar la pagina con el formato deseado
\fancyhead{}
\fancyhead[L]{\footnotesize{Sistemas Dinámicos}}  
\fancyhead[C]{Corto 1}
\fancyhead[R]{\footnotesize{\thepage}}
%\fancyhead[LO,RE]{Cálculo 3}
%\fancyhead[RO,LE]{\footnotesize{\thepage}}
\fancyfoot{}
%\fancyfoot[L]{Diego Sarceño}
%\fancyfoot[LO,RE]{Diego Sarceño}
%%%%%%%%%%%%%%%%%%%%%%%%%%%%%%%%%%%%%%%%%%%%%%%%%%%%%%%%%%%
%% NUEVA BARRA INFERIOR, NICEEEE :3
\usepackage{fourier-orns}

\renewcommand\footrule{%
\hrulefill
\raisebox{-2.1pt}
{\quad\decosix\quad}%
\hrulefill}
%%%%%%%%%%%%%%%%%%%%%%%%%%%%%%%%%%%%%%%%%%%%%%%%%%%%%%%%%%%
\newcommand{\inner}[2]{\langle #1 , #2 \rangle}
\newcommand{\metric}[2]{\rho(#1,#2)}	
\newcommand{\seque}[2]{\{ #1_{#2} \}}
%%%%%%%%%%%%%%%%%%%%%%%%%%%%%%%%%%%%%%%%%%%%%%%%%%%%%%%%%%%
\definecolor{DS_Black}{HTML}{000000}

\begin{document}
\begin{titlepage}
\input{Header_original}

%\noindent \textbf{Instrucciones: } Resuelva cada uno de los siguientes problemas a \LaTeX  o a mano con letra clara y legible, dejando constancia de sus procedimientos. No es necesaria la carátula, únicamente su identificaciónn y las respuestas encerradas en un cuadro.

\vspace{1cm}


\begin{table}[H]
	\centering
	\begin{tabular}{||c|p{13cm}||}
		\hline
		\hline
			No. & Temas \\
		\hline
		\hline
			1 & Programa del curso. \\
			2 & Definiciones iniciales. \\
			3 & Ejemplos y líneas de fase. \\
			4 & Propiedades de los SD. \\
			5 & Recurrencias. \\
			6 & Dinpamicas simples, introducción. \\
			7 & Topología de espacios métricos. \\
			8 & Hoja de trabajo 1. \\
			9 & Discusión y dudas. \\
			10 & Lipschitz, determinismo y contracciones. \\
			11 & El principio de contracción. \\
			12 & Ejemplo en 2D (contracción). \\
			13 & Localizando puntos fijos, y Herón. \\
			14 & Estabilidad y telarañas. \\
			15 & Examen corto 1. \\
			16 & Homeomorfismos, conjugaciones y equivalencias. \\
			17 & Invariantes, $\alpha$ y $\omega$ límites, cuencas. \\
			18 & Mapas crecientes, perturbaciones y bifurcaciones. \\
			19 & Tipos de bifurcación y tests. \\
			20 & El mapa logístico. \\
			21 & Ejercicios y dudas. \\
			22 & Dinámica en wolfram. \\
			23 & Discretización y sistemas análogos. \\
			24 & Aplicaciones de la dinámica. \\
			25 & Resolución del corto 1. \\
			26 & Variedades. \\
			27 & Diagonalización y forma de Jordan. \\
			28 & Discusión. \\
			29 & Fractales. \\
			30 & Videos 2. \\
			31 & Sistemas de EDOs. \\
			32 & Ejemplos de Sistemas. \\
			33 & Primer parcial. \\
		\hline
		\hline
	\end{tabular}
\end{table}



\begin{table}[H]
	\centering
	\begin{tabular}{||c|p{13cm}||}
		\hline
		\hline
			No. & Temas \\
		\hline
		\hline
			34 & La exponencial de una matriz. \\
			35 & Diagramas de fase 2D, parte 1. \\
			36 & Dudas de tareas. \\
			37 & Diagramas de fase 2D, parte 2. \\
			38 & Diagramas de fase 2D, parte 3. \\
			39 & Videos 3, más aplicaciones. \\
			40 & Diagrama de Poincaré, mapas lineales y logaritmos. \\
			41 & Discusión. Temas Proyectos. \\
			42 & Asignación del Examen Corto. \\
			43 & Sistemas no lineales y linealización. \\
			44 & Poincaré-Lyapunov, Hartman-Grobman y Bendixon. \\
			45 & Poincaré-Bendixon. \\
			46 & Principio de las Casillas. \\
			47 & Recurrencia y rotaciones. \\
			48 & Benford y densidad de órbitas. \\
			49 & Demostración de la densidad de órbitas. \\
			50 & Discusión. Equidistribución. \\
			51 & Demostración de la equidistribución. \\
			52 & Examen parcial 2. \\
			53 & Equidistribución vía Fourier. \\ 
			54 & Flujos y traslaciones torales. \\
			55 & Preservación de volumen para mapas. \\
			56 & Preservación de volumen para flujos. \\
			57 & Sistemas Newtonianos, introducción. \\
			58 & Mecánica de Lagrange. \\
			59 & Exposiciones 1. \\
			60 & Exposiciones 2. \\
		\hline
		\hline
	\end{tabular}
\end{table}





%%%%





%\begin{thebibliography}{00}
%\bibitem{b1} Falomir, H. (2015). \textit{Curso de métodos de la física matemática.} Series: Libros de Cátedra.
%\bibitem{b2} Saxe, K. (2002). \textit{Beginning functional analysis (p. 7)}. New York: Springer.
%\bibitem{b3} Reed, M. (2012). \textit{Methods of modern mathematical physics: Functional analysis.} Elsevier.
%\bibitem{b4} Axler, S. (2015). \textit{Linear algebra done right.} springer publication.
%\bibitem{b5} Bartle, R. G., \& Sherbert, D. R. (2000). \textit{Introduction to real analysis.} John Wiley \& Sons, Inc..
%\end{thebibliography}



\end{titlepage}
\end{document}
